\documentclass[12pt]{article}
\usepackage[portuguese, english]{babel}
\usepackage[T1]{fontenc}
\usepackage[a4paper, left=2.5cm, top=2.5cm]{geometry}
\usepackage{syntax}
\usepackage{authblk}
\usepackage{fontspec}
\usepackage{graphicx}
\usepackage{hyperref}
\usepackage{setspace}
\usepackage{subcaption}
\usepackage{verbatimbox}
\usepackage[dvipsnames]{xcolor}
\usepackage{url}
\usepackage{multicol}

\newcommand\myshade{85}
\newcommand{\pprime}{\ensuremath{^{\prime}}}
\addto\captionsportuguese{\renewcommand*\contentsname{Sumário}}
\setcounter{secnumdepth}{0}
\colorlet{mylinkcolor}{violet}
\colorlet{mycitecolor}{YellowOrange}
\colorlet{myurlcolor}{Aquamarine}
\onehalfspacing

\hypersetup{
  linkcolor  = mylinkcolor!\myshade!black,
  citecolor  = mycitecolor!\myshade!black,
  urlcolor   = myurlcolor!\myshade!black,
  colorlinks = true,
}

\title{Texto auxiliar para Atividade Avaliativa \#2: Analisador Léxico e Analisador Sintático}
\author[1]{Pedro Santi Binotto [20200634]\thanks{\texttt{pedro.binotto@grad.ufsc.br}}}
\author[2]{Eduardo Caigar Dudel [18206076]\thanks{\texttt{dudeleduardo@gmail.com}}}
\author[3]{Arthur Alexandre Nascimento [18200410]\thanks{\texttt{arthuralexnascimento@hotmail.com}}}
\author[4]{Eduardo Sousa Szczepaniak [20202478]\thanks{\texttt{szcz.eduardo@gmail.com}}}
\date{\today}
\affil[1]{Departamento de Informática e Estatística, Universidade Federal de Santa Catarina}

\begin{document}
\begin{titlepage}
\selectlanguage{portuguese}
\maketitle
\thispagestyle{empty}

\begin{abstract}
  Relatório do projeto prático da disciplina de \textit{Introdução a Compiladores}.
\end{abstract}

\end{titlepage}

\newpage
\tableofcontents

\newpage
\section{Elaboração do trabalho}
\paragraph{}
O trabalho foi elaborado utilizando as ferramentas \textit{Flex}\cite{mitFlex1990} e \textit{Bison}\cite{mitBison1992}
para gerar os analisadores léxico e sintático, respectivamente. Além dos manuais das ferramentas, a apostila
``\textit{Lex \& Yacc}''\cite{niemannYacc} foi outro recurso extensivamente consultado para utilização das tecnologias
empregadas.

Para gerar o analisador sintático, uma descrição idêntica à especificação encontrada no documento da proposta do
trabalho foi fornecida ao \textit{Bison} para conversão. 

Caso desejássemos construir uma \textit{parse-table} LL(1), seria necessário adequar a escrita da sintaxe para
possibilitar a análise \textit{top-down sem backtracking}; entretanto, não é possível remover todos os casos de
ambiguidades na gramática sem fazer alterações na especificação da linguagem, como a forma das cláusulas
``\texttt{if...else}'' é inerentemente ambígua. 

Uma versão da gramática ajustada para resolver os casos de ambiguidade possíveis e livre de recursão à esquerda pode ser
encontrada na seção ``\nameref{sec:normalizacao-gramatica}''.

A implementação final dos analisadores adiciona uma propriedade além do que descreve a especificação da gramática para
possibilitar a escrita de comentários nos arquivos de código fonte:

\begin{verbatim}
/*
 * Texto dentro destes delimitadores não será interpretado.
 *
 * Palavras-chave e símbolos da linguagem também não produzem
 * efeito nenhum:
 *
 * def int print return () {} ;
 */

def main() {
  int A = calc_a();
  return;
}
\end{verbatim}

Os arquivos de teste utilizam destes comentários como uma forma de registrar metadados sobre o caso de teste e saída
esperada da execução da análise.

\newpage
\section{Normalização da gramática}
\label{sec:normalizacao-gramatica}
\subsection{Gramática original da linguagem}

\paragraph{}
A seguir está a gramática original, como fornecida na proposta do trabalho, em notação BNF:

\begin{multicols}{2}
\raggedcolumns
\setlength{\columnseprule}{0.2pt}
\begin{grammar}
<main> ::= <stmt>
  \alt <flist>
  \alt <empty>

<flist> ::= <fdef> <flist>
  \alt <fdef>

<fdef> ::= ‘\texttt{def}’ ‘\texttt{id}’ ‘\texttt{(}’ <parlist> ‘\texttt{)}’ ‘\texttt{\{}’ <stmtlist> ‘\texttt{\}}’

<parlist> ::= ‘\texttt{int}’ ‘\texttt{id}’ ‘\texttt{,}’ <parlist>
  \alt ‘\texttt{int}’ ‘\texttt{id}’
  \alt <empty>

<stmt> ::= ‘\texttt{int}’ ‘\texttt{id}’ ‘\texttt{;}’
  \alt <atribst> ‘\texttt{;}’
  \alt <printst> ‘\texttt{;}’
  \alt <returnst> ‘\texttt{;}’
  \alt <ifstmt>
  \alt ‘\texttt{\{}’ <stmtlist> ‘\texttt{\}}’
  \alt ‘\texttt{;}’

<atribst> ::= ‘\texttt{id}’ = <expr>
  \alt ‘\texttt{id}’ = <fcall>

<fcall> ::= ‘\texttt{id}’ ‘\texttt{(}’ <parlistcall> ‘\texttt{)}’

<parlistcall> ::= ‘\texttt{id}’ ‘\texttt{,}’ <parlistcall>
  \alt ‘\texttt{id}’
  \alt <empty>

<printst> ::= ‘\texttt{print}’ <expr>

<returnst> ::= ‘\texttt{return}’

<ifstmt> ::= ‘\texttt{if}’ ‘\texttt{(}’ <expr> ‘\texttt{)}’ <stmt> ‘\texttt{else}’ <stmt>
  \alt ‘\texttt{if}’ ‘\texttt{(}’ <expr> ‘\texttt{)}’ <stmt>

<stmtlist> ::= <stmt> <stmtlist>
  \alt <stmt>

<expr> ::= <numexpr> ‘\texttt{\char60}’ <numexpr>
  \alt <numexpr> ‘\texttt{>}’ <numexpr>
  \alt <numexpr> ‘\texttt{==}’ <numexpr>
  \alt <numexpr>

<numexpr> ::= <numexpr> ‘\texttt{+}’ <term>
  \alt <numexpr> ‘\texttt{-}’ <term>
  \alt <term>

<term> ::= <term> ‘\texttt{*}’ <factor>
  \alt <factor>

<factor> ::= ‘\texttt{num}’
  \alt ‘\texttt{(}’ <numexpr> ‘\texttt{)}’
  \alt ‘\texttt{id}’
\end{grammar}
\end{multicols}

\newpage
\subsection{Gramática ajustada}
\paragraph{}
Gramática ajustada, livre de recursão à esquerda e corrigida para retificar todos os casos de ambiguidade que são 
possíveis; em notação BNF:

\begin{multicols}{2}
\raggedcolumns
\setlength{\columnseprule}{0.2pt}
\begin{grammar}
<main> ::= <stmt> 
  \alt <flist> 
  \alt <empty>

<flist> ::= <fdef> <flist \pprime>

<flist \pprime> ::= <flist> 
  \alt <empty>

<fdef> ::= ‘\texttt{def}’ ‘\texttt{id}’ ‘\texttt{(}’ <parlist> ‘\texttt{)}’ ‘\texttt{\{}’ <stmtlist> ‘\texttt{\}}’

<parlist> ::= ‘\texttt{int}’ ‘\texttt{id}’ <parlist \pprime>

<parlist \pprime> ::= ‘\texttt{,}’ <parlist> 
  \alt <empty>

<stmt> ::= ‘\texttt{int}’ ‘\texttt{id}’ ‘\texttt{;}’
  \alt <atribst> ‘\texttt{;}’
  \alt <printst> ‘\texttt{;}’
  \alt <returnst> ‘\texttt{;}’
  \alt <ifstmt>
  \alt ‘\texttt{\{}’ <stmtlist> ‘\texttt{\}}’
  \alt ‘\texttt{;}’

<atribst> ::= ‘\texttt{id}’ = <atribst \pprime>

<atribst \pprime> ::= <expr> 
  \alt <fcall>

<fcall> ::= ‘\texttt{id}’ ‘\texttt{(}’ <parlistcall> ‘\texttt{)}’

<parlistcall> ::= ‘\texttt{id}’ <parlistcall \pprime>

<parlistcall \pprime> ::= ‘\texttt{,}’ <parlistcall> 
  \alt <empty>

<printst> ::= ‘\texttt{print}’ <expr>

<returnst> ::= ‘\texttt{return}’

<ifstmt> ::= ‘\texttt{if}’ ‘\texttt{(}’ <expr> ‘\texttt{)}’ <stmt> <ifstmt \pprime>

<ifstmt \pprime> ::= ‘\texttt{else}’ <stmt> 
  \alt <empty>

<stmtlist> ::= <stmt> <stmtlist \pprime>

<stmtlist \pprime> ::= <stmtlist> 
  \alt <empty>

<expr> ::= <numexpr> <expr \pprime>

<expr \pprime> ::= ‘\texttt{\char60}’ <numexpr>
  \alt ‘\texttt{>}’ <numexpr>
  \alt ‘\texttt{==}’ <numexpr>
  \alt <empty>

<numexpr> ::= <term> <numexpr \pprime>

<numexpr \pprime> ::= ‘\texttt{+}’ <term> <numexpr \pprime>
  \alt ‘\texttt{-}’ <term> <numexpr \pprime>
  \alt <empty>

<term> ::= <factor> <term \pprime>

<term \pprime> ::= ‘\texttt{*}’ <factor> <term \pprime>
  \alt <empty>

<factor> ::= ‘\texttt{num}’
  \alt ‘\texttt{(}’ <numexpr> ‘\texttt{)}’
  \alt ‘\texttt{id}’
\end{grammar}
\end{multicols}

\newpage
\section{Referências Bibliográficas}
\bibliographystyle{plain}
\bibliography{references}

\end{document}
